\documentclass[../main.tex]{subfiles}
\graphicspath{{\subfix{../images/}}}

\begin{document}

Demuéstrese o refútese razonadamente la equivalencia entre los pares de enunciados siguientes:

\subsection{Primer par de enunciados}
\begin{itemize}
    \item $\neg \exists x : \neg(\neg p(x)\lor \neg q(x))$, al que llamaremos primer enunciado
    \item $[\forall x, \neg p(x)] \lor [\forall x, \neg q(x)]$, al que llamaremos segundo enunciado
\end{itemize}

\bigskip
\textbf{Solución:}

\begin{enumerate}
    \item Para demostrar o refutar su equivalencia partiremos aplicando al primer enunciado la siguiente ley de Morgan:
    \begin{equation*}
        \forall x,  p(x) \equiv \neg [\exists x : \neg p(x)]
    \end{equation*}
    
    \item De esta forma aplicando la ley descrita en el paso anterior sobre el primer enunciado $\neg \exists x : \neg(\neg p(x)\lor \neg q(x))$, podríamos también escribirlo así:
    \begin{equation*}
        \forall x : \neg p(x) \lor \neg q(x)
    \end{equation*}

    \item Además sabemos que existe una propiedad de la equivalencia que nos dice: 
    \begin{equation*}
        [\forall x, p(x)] \lor [\forall x, q(x)] \models \forall x : [p(x) \lor q(x)]
    \end{equation*}
    
    \item Tomando esto último en consideración podríamos aplicarlo al primer enunciado y quedaría demostrado que: 
    \begin{equation*}
        [\forall x, \neg p(x)] \lor [\forall x, \neg q(x)] \models \forall x : [\neg p(x) \lor \neg q(x)]
    \end{equation*}

    \item Por esto último podemos concluir que el primer enunciado implica al segundo, pero no en sentido contrario, por lo tanto, \textbf{NO son equivalentes}, el primer enunciado implica el segundo pero el segundo no implica el primero.
\end{enumerate}

%------------------------------------------------

\subsection{Segundo par de enunciados}
\begin{itemize}
    \item $\neg ( \forall x, \exists y: [ (p(x,y) \land q(x,y)) \implies r(x,y) ] )$, al que llamaremos primer enunciado
    \item $\exists x : [(\forall y, p(x,y)) \land (\forall y, q(x,y)) \land (\forall y, \neg r(x,y))]$, al que llamaremos segundo enunciado
\end{itemize}

\bigskip
\textbf{Solución:}

\begin{enumerate}
    \item Analizando inicialmente ambos enunciados decido partir por el segundo, intentando resolver primero la parte interna del mismo. A esta parte la llamaremos AUX, de esta forma podríamos reescribir el segundo enunciado como:
    \begin{equation*}
        \exists x : AUX,
    \end{equation*}
    donde AUX = $[(\forall y, p(x,y)) \land (\forall y, q(x,y)) \land (\forall y, \neg r(x,y))]$
    
    \item Ahora nos concentraremos en simplificar AUX y para ello aplicaremos la siguiente propiedad:
    \begin{equation*}
        [\forall x, p(x)] \land [\forall x, q(x)] \equiv \forall x : [p(x) \land q(x)]
    \end{equation*}

    \item Aplicando lo anterior sobre AUX = $[(\forall y, p(x,y)) \land (\forall y, q(x,y)) \land (\forall y, \neg r(x,y))]$, AUX nos quedaría de la siguiente forma: 
    \begin{equation*}
        \forall y, [p(x,y)) \land q(x,y) \land \neg r(x,y)]
    \end{equation*}
    
    \item Además sabemos que existe una ley de Morgan que nos dice:
    \begin{equation*}
        \forall x,  p(x) \equiv \neg [\exists x : \neg p(x)]
    \end{equation*}
    
    \item Si aplicamos esta ley sobre AUX, tendríamos que AUX sería igual a: 
    \begin{equation*}
        \neg \exists y, \neg [p(x,y)) \land q(x,y) \land \neg r(x,y)]
    \end{equation*}
    
    \item También sabemos que:
    \begin{equation*}
        a \land (b \land c) \equiv (a \land b) \land c
    \end{equation*}
    y que 
    \begin{equation*}
        \neg (a \land b) \equiv \neg a \lor \neg b
    \end{equation*}
    
    \item Si consideramos:
    \begin{equation*}
        a = p(x,y)) \land q(x,y),
    \end{equation*}
    \begin{equation*}
        b= \neg r(x,y)
    \end{equation*}
    
    \item Podríamos perfectamente reescribir AUX como: 
    \begin{equation*}
        \neg \exists y, [\neg (p(x,y)) \land q(x,y)) \lor \neg(\neg r(x,y))]
    \end{equation*}
    o lo que es lo mismo:
    \begin{equation*}
        \neg \exists y, [\neg (p(x,y)) \land q(x,y)) \lor r(x,y)]
    \end{equation*}
    
    \item Para finalizar con la manipulación de AUX, también recordemos que $\neg p \lor q \equiv p \implies q$, y aplicando esto sobre AUX quedaría de la siguiente forma:
    \begin{equation*}
        \neg \exists y, [(p(x,y)) \land q(x,y)) \implies r(x,y)]
    \end{equation*}
    
    \item Ahora podemos sustituir AUX en su original, es decir, en el segundo enunciado:
    \begin{equation*}
        \exists x : AUX,
    \end{equation*}
    \begin{equation*}
        \exists x : \neg \exists y, [(p(x,y)) \land q(x,y)) \implies r(x,y)]
    \end{equation*}
    
    \item Para finalizar, aplicaremos sobre el cuantificador existencial de x la siguiente ley:
    \begin{equation*}
        \exists x : p(x) \equiv \neg[\forall x, \neg p(x)]
    \end{equation*}
    
    \item Quedando el segundo enunciado reescrito de la siguiente forma:
    \begin{equation*}
        \neg ( \forall x, \exists y: [ (p(x,y) \land q(x,y)) \implies r(x,y) ] )
    \end{equation*}
    
    \item lo que es igual al primer enunciado, por lo tanto estos dos enunciados \textbf{SON EQUIVALENTES}
    
\end{enumerate}

\end{document}